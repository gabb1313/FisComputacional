\documentclass[12pt]{article}
\usepackage[spanish,mexico]{babel}
	\selectlanguage{spanish}
\usepackage{graphicx}
\usepackage{amsmath}
\usepackage{wrapfig}
\usepackage{float}
\usepackage[utf8]{inputenc}


\title{Actividad 2: Elementos de la programación Python 1}
\author{Ana Gabriela Carretas Talamante}
\date{27 de enero de 2016}

\begin{document}
\maketitle
\section{Introducción}
Python es un lenguaje de programación interpretado cuya filosofía hace hincapié en una sintaxis que favorezca un código legible \cite{1}. En esta actividad se procuraron códigos en lenguaje Python con la misión de mostrar el uso de las variables y la redacción en el lenguaje designado. Se presentarán entonces las modificaciones realizadas a cada ejemplo de programa, sugeridos en la página de la materia \cite{2}.

\section{Problema 1}
``Se deja caer una pelota desde el techo de una torre de altura $h$. Se desea saber la altura de la pelota respecto a la torre a un determinado tiempo después de haber sido dejada caer."

\subparagraph*{Programa original: caida.py}
\begin{verbatim}
h = float(input("Proporciona la altura de la torre: "))
t = float(input("Ingresa el tiempo: "))
s = 0.5*9.81*t**2
print("La altura de la pelota es", h-s, "metros")
-----------
Proporciona la altura de la torre: 10
Ingresa el tiempo: 1.42
('La altura de la pelota es', 0.10955799999999982, 'metros')
\end{verbatim}

Modificando el código anterior, se obtuvo uno en el que tras pedir la altura $h$ en metros, imprimiera el tiempo en segundos que tarda en llegar la pelota al suelo.

\subparagraph*{Programa modificado}
\begin{verbatim}
from math import sqrt
h = float(input("Proporciona la altura de la torre: "))
t = sqrt(2*h/9.8)
print("La pelota tarda en caer", t, "segundos")
-----------
Proporciona la altura de la torre: 10
('La pelota tarda en caer', 1.4285714285714286, 'segundos')
\end{verbatim}

\section{Problema 2}
``Un satélite orbita la Tierra a una altura $h$, con un periodo $T$ en segundos. Demuestre que la altitud $h$ del satélite sobre la superficie de la Tierra esta dado por la expresión:
\begin{equation} \label{1}
(R+h)^3=\frac{GMT^2}{4\pi^2}
\end{equation}
donde $G=6.67 \times 10^{-11} \frac{m^3}{kg s^2}$ es la constante de Gravitación Universal de Newton, $M=5.97 \times 10^{24} kg$ es la masa de la Tierra y $R=6371 km$ su radio."

\subparagraph*{Demostración}
Suponiendo una órbita circular, tendremos un satélite que orbite a velocidad constante, aplicando la Segunda Ley de Newton, considerando que la aceleración sería centrípeta:
\begin{equation} \label{2}
F=m\frac{v^2}{r}
\end{equation}
Siendo $F$ la fuerza gravitatoria entre la Tierra y el satélite:
\begin{equation} \label{3}
F=\frac{GMm}{r^2}
\end{equation}
Expresamos a la velocidad en términos del período del satélite:
\begin{equation} \label{4}
v=\frac{2\pi r}{T}
\end{equation}
Sustituyendo $v$ en \eqref{2} e igualando con \eqref{3} obtenemos:
\begin{equation} \label{5}
\frac{GMm}{r^2}=m\frac{{(\frac{2\pi r}{T})}^2}{r}
\end{equation}
En el caso más general de un satélite orbitando un planeta, la altura de su órbita no es despreciable respecto al radio de la órbita, por lo que la distancia $r$ es igual a la suma del radio del planeta $R$ más la altura $h$ de la órbita sobre su superficie \cite{3}. Por lo que $r=(R+h)$. Simplificando \eqref{5} y sustituyendo a $r$, llegamos a \eqref{1}:
\begin{equation}
(R+h)^3=\frac{GMT^2}{4\pi^2}
\end{equation}

Se escribió un programa donde imprimía la altura $h$ en metros del satélite con órbita de período $T$, el cual se pide al usuario. 

\subparagraph*{Programa creado}
\begin{verbatim}
from math import pi
T = float(input("Proporciona el período de órbita en segundos: "))
k = (6.67e-11*5.97e24)/(4*pi*pi)
h = (k*T*T)**(1./3.)-6371000
print("La altura del objeto con respecto a la superficie terrestre es",
h, "metros")
\end{verbatim}

Este arrojó resultados para los tiempos:
\begin{itemize}
\item \textbf{24 horas:}
\begin{verbatim}
Proporciona el período de órbita en segundos: 86400
('La altura del objeto con respecto a la superficie terrestre es',
35855910.176174976, 'metros')
\end{verbatim}
\item \textbf{90 minutos:}
\begin{verbatim}
Proporciona el período de órbita en segundos: 5400
('La altura del objeto con respecto a la superficie terrestre es',
279321.6253728606, 'metros')
\end{verbatim}
\item \textbf{45 minutos:}
\begin{verbatim}
Proporciona el período de órbita en segundos: 2700
('La altura del objeto con respecto a la superficie terrestre es',
-2181559.8978108233, 'metros')
\end{verbatim}
\end{itemize}

\section{Problema 3}
``Un punto en el espacio en el sistema de coordenadas polares se describe por las cantidades $(r, \theta)$. La relación entre coordenadas polares y el sistema de coordenadas cartesianos, esta dada por las ecuaciones:  $x = r cos{\theta}, y = r sin{\theta}$. Ingrese el siguiente programa para calcular las coordenadas cartesianas a partir de las coordenadas polares. "

\subparagraph*{Programa original: Polar.py}
\begin{verbatim}
from math import sin,cos,pi
r = float(input("Introduce r: "))
d = float(input("Ingresa theta en grados: "))
theta = d*pi/180
x = r*cos(theta)
y = r*sin(theta)
print("x =",x," y =",y)
-----------
Introduce r: 5
Ingresa theta en grados: 30
x = 4.330127018922194  y = 2.4999999999999996
\end{verbatim}

Se creó un programa que a partir de introducir coordenadas rectangulares $(x,y,z)$, imprimiera esféricas $(\rho,\theta,\phi)$.

\subparagraph*{Programa creado}
\begin{verbatim}
from math import atan,acos,pi,sqrt
x = float(input("Introduce x: "))
y = float(input("Introduce y: "))
z = float(input("Introduce z: "))
rho = sqrt((x*x)+(y*y)+(z*z))
theta = atan(y/x)
phi = acos(z/(sqrt((x*x)+(y*y)+(z*z))))
print("rho =",rho," theta =",theta," phi =",phi)
-----------
Introduce x: 4
Introduce y: 6
Introduce z: 3
rho = 7.810249675906654  theta = 0.982793723247329  phi = 1.176551951730111
\end{verbatim}

\section{Problema 4}
``Números pares (even) e impares(odd). Sabemos que los números pares son divisibles entre 2, es decir que su residuo es cero $(2n)$, mientras que los impares tienen residuo 1 $(2n+1)$. Nos apoyamos en la operación de módulo \%."

\begin{verbatim}
n = int(input("Enter an integer: "))
if n%2==0:
     print("even")
else:
     print("odd")
-----------     
Enter an integer: 6
even
\end{verbatim}

``Ingresa el siguiente código que admite 2 enteros, que se utiliza el control \textsc{while} (mientas que). Comprende su dinámica."

\subparagraph*{Programa original: EvenOdd.py}
\begin{verbatim}
print("Enter two integers, one even, one odd.")
m = int(input("Enter the first integer: "))
n = int(input("Enter the second integer: "))
while (m+n)%2==0:
    print("One must be even and the other odd.")
    m = int(input("Enter the first integer: "))
    n = int(input("Enter the second integer: "))
print("The numbers you chose are",m,"and",n)
-----------
Enter two integers, one even, one odd.
Enter the first integer: 5
Enter the second integer: 5
One must be even and the other odd.
Enter the first integer: 5
Enter the second integer: 6
The numbers you chose are 5 and 6
\end{verbatim}

\section*{Problema 5}
``Los Números de Fibonacci  es una sucesión de números enteros aparecen en toda la naturaleza. El siguiente programa calcula la secuencia de Fibonacci, introduce la condición de control \textsc{while} (mientras que)."

\subparagraph*{Programa original: Fibonacci.py}
\begin{verbatim}
f1,f2 = 1,1
while f2<1000:
      print(f2)
      f1,f2 = f2,f1+f2
-----------     
1
2
3
5
8
13
21
34
55
89
144
233
377
610
987
\end{verbatim}

Basándonos en la idea de el programa de la serie se Fibonacci, se creó uno para la secuencia de números de Catalan, dados por la fórmula de recurrencia:
\begin{equation*}
C_0=1, C_{(n+1)} = \frac{2(2n+1)}{(n+2)}C_n
\end{equation*}

\pagebreak

\subparagraph*{Programa creado}
\begin{verbatim}
n,c = 0,1
while c<100000:
      print(c)
      n,c = n+1,(((4*n)+2)/(n+2))*c
-----------
1
1.0
2.0
5.0
14.0
42.0
132.0
429.0
1430.0
4862.0
16796.0
58786.0
\end{verbatim}


\begin{thebibliography}{6}

\bibitem{1}
Wikipedia,
\emph{Python}. Recuperado el 27 de enero de 2016 de https://es.wikipedia.org/wiki/Python

\bibitem{2}
Lizárraga, C.
\emph{Actividad 2 (2016-1)}. Recuperado el 27 de enero de 2016 de http://computacional1.pbworks.com/w/page/104476954/Actividad\%202\%20(2016-1)

\bibitem{3}
\emph{Gravitación y tercera Ley de Kepler}. Recuperado el 27 de enero de 2016 de
http://e-ducativa.catedu.es/44700165/aula/archivos/repositorio/3000/3227/html/22\_gravitacin\_y
\_tercera\_ley\_de\_kepler.html

\end{thebibliography}

\end{document}

